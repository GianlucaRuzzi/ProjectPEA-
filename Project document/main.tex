\documentclass[11pt,twoside]{article}
\usepackage[utf8]{inputenc}
\usepackage[T1]{fontenc}
\usepackage{url}
\usepackage{titlesec}
\usepackage{graphicx} % opzionale per immagini o decorazioni
\usepackage{titlesec} % Pacchetto per decorare i titoli

\usepackage{listings}

\lstdefinelanguage{Alloy}{
    keywords={abstract, but, extends, iff, lone, one, set, all, check, fact, implies, module, open, sig, and, disj, for, in, no, or, some, as, else, fun, Int, none, pred, sum, assert, exactly, iden, let, not, run, univ},
    sensitive=true,
    comment=[l]{//},         
    morecomment=[s]{/*}{*/}, 
    morestring=[b]",         
}

\lstset{
    basicstyle=\small\ttfamily,          
    keywordstyle=\color{blue},     
    commentstyle=\color{green},     
    stringstyle=\color{red}, 
    tabsize = 2,
    xleftmargin = 0cm,
    xrightmargin = 0cm,
    language=Alloy                 
}

\titleformat{\section} % Stile della section
  {\normalfont\huge\bfseries} % Stile del testo: font normale, LARGE (dimensione), grassetto
  {\thesection} % Mostra il numero della section
  {1em} % Spaziatura tra numero e titolo
  {} % Cosa appare prima del titolo (vuoto in questo caso)


\renewcommand{\thesubsection}{\Alph{subsection}} % Numerazione con lettere maiuscole
\titleformat{\subsection}[hang]
{\normalfont\LARGE\bfseries} % Stile del titolo della subsection
{\thesubsection} % Visualizza la lettera
{1em}{} % Distanza tra la lettera e il titolo


\renewcommand{\thesubsubsection}{\thesubsection.\arabic{subsubsection}} % Numerazione A.1
\titleformat{\subsubsection}[hang]
{\normalfont\Large\bfseries} % Stile del titolo della subsubsection
{\thesubsubsection} % Visualizza il numero A.1
{1em}{} % Distanza tra il numero e il titolo

%% !TEX root = main.tex
% Page margins, header and footer positions
\usepackage{geometry}
\geometry{
 a4paper,
 total={210mm,297mm},
 left=25mm,
 right=25mm,
 top=30mm,
 bottom=25mm,
 headsep=7mm
}

\interfootnotelinepenalty=10000

% To display filling dots in the TOC for all entries
\usepackage[titles]{tocloft}
\renewcommand{\cftsecleader}{\cftdotfill{\cftdotsep}}

% Define new header and footer style #in each page
\usepackage{fancyhdr}

\pagestyle{fancy}
\fancyhf{}
\lhead{\color{Gray}{\small{Project by Gianluca Ruzzi}}}
\lfoot{\textcolor{Gray}{\small{Copyright © 2025, Gianluca Ruzzi – All rights reserved}}}
\rfoot{\textcolor{Gray}{\thepage}}
\renewcommand{\headrulewidth}{0pt}

% PACKAGES
\usepackage{wasysym}
\usepackage{pifont}

\newcommand{\supported}{\ding{52}\xspace}
\newcommand{\unsupported}{\ding{55}\xspace}
\newcommand{\partsupported}{\textcolor{black!40}{\ding{52}}\xspace}
\newcommand{\lowsupported}{\textcolor{black!20}{\ding{52}}\xspace}
\newcommand{\unknowsupported}{\textbf{?}\xspace}

% Font: Times
\usepackage{times}
% Change monospaced font
\renewcommand{\ttdefault}{lmtt}

% Tables
\usepackage{tabu}
\usepackage{tabularx}
\usepackage{ltablex}
\usepackage{longtable}
\usepackage{float}

% Landscape mode
\usepackage{pdflscape}
\usepackage{rotating}
\usepackage{caption}

% Make landscape mode be sensitive to even and odd pages
\def\myrotate{\ifodd\c@page\else-\fi 90}
\makeatletter
\global\let\orig@begin@landscape=\landscape%
\global\let\orig@end@landscape=\endlandscape%
\gdef\@true{1}
\gdef\@false{0}
\gdef\landscape{%
    \global\let\within@landscape=\@true%
    \orig@begin@landscape%
}%
\gdef\endlandscape{%
    \orig@end@landscape%
    \global\let\within@landscape=\@false%
}%
\@ifpackageloaded{pdflscape}{%
    \gdef\pdf@landscape@rotate{\PLS@Rotate}%
}{
    \gdef\pdf@landscape@rotate#1{}%
}
\let\latex@outputpage\@outputpage
\def\@outputpage{
    \ifx\within@landscape\@true%
        \if@twoside%
            \ifodd\c@page%
                \gdef\LS@rot{\setbox\@outputbox\vbox{%
                    \pdf@landscape@rotate{-90}%
                    \hbox{\rotatebox{90}{\hbox{\rotatebox{180}{\box\@outputbox}}}}}%
                }%
            \else%
                \gdef\LS@rot{\setbox\@outputbox\vbox{%
                    \pdf@landscape@rotate{+90}%
                    \hbox{\rotatebox{90}{\hbox{\rotatebox{0}{\box\@outputbox}}}}}%
                }%
            \fi%
        \else%
            \gdef\LS@rot{\setbox\@outputbox\vbox{%
                \pdf@landscape@rotate{+90}%
                \hbox{\rotatebox{90}{\hbox{\rotatebox{0}{\box\@outputbox}}}}}%
            }%
        \fi%
    \fi%
    \latex@outputpage%
}
\makeatother

% Graphics
\usepackage{graphicx}
\usepackage[dvipsnames, table]{xcolor}


% Other
\usepackage{ifthen}
\usepackage{xspace}
\usepackage{enumitem}
\usepackage{amssymb}
\usepackage[pdftex, colorlinks]{hyperref}
\newcommand{\comment}[1]{{\color{Red}$\blacktriangleright$ Comment: #1 $\blacktriangleleft$}}


% Some utilities
\input{util/highlight}
\input{util/abbrev}

\begin{document}

% Definizione dei colori
\definecolor{darkblue}{RGB}{0,51,102} % Blu scuro

% Personalizzazione delle sezioni
\titleformat{\section} % Formato delle sezioni
  {\normalfont} % Stile di base
  {\color{darkblue}\fontsize{40}{40}\bfseries\thesection\ |} % Numero sezione e linea più grandi
  {0.3em} % Spaziatura tra numero e titolo
  {\color{darkblue}\normalfont\Huge\bfseries} % Titolo in blu scuro
  [\vspace{0.5em}\titlerule] % Linea decorativa sotto il titolo

% Personalizzazione delle subsections
\titleformat{\subsection} % Sottosezioni
  {\normalfont\color{darkblue}\LARGE\bfseries} % Stile del testo: blu, grande e grassetto
  {\thesubsection} 
  {1em} % Spaziatura tra numero e titolo
  {} % Testo senza decorazioni
  [] % Nessuna decorazione aggiuntiva

\definecolor{mediumblue}{RGB}{51,102,153} % Blu più chiaro per subsubsections

% Personalizzazione delle subsubsections
\titleformat{\subsubsection} % Subsubsections
  {\normalfont\color{mediumblue}\large} % Blu più chiaro e stile meno grassetto
  {\thesubsubsection} 
  {1em} % Spaziatura
  {} % Nessuna decorazione aggiuntiva
  []
  
% TITLE PAGE
\begin{titlepage}
% LOGO
{\begin{table}[t!]
\centering
\begin{tabu} to \textwidth { X[1.0,r,p] X[1.7,l,p] }
\textcolor{Blue}
{\textcolor{darkblue}{\textbf{\small{Performance Evaluation and Application project - 2024/2025}}}} & \includegraphics[scale=0.8]{images/PolimiLogo}
\end{tabu}
\end{table}}~\\ [4 cm]


% TITLE 
\begin{center} %centered
{\textcolor{darkblue}{\textbf{\Huge{Executing AI algorithm with a coprocessor}}}} \\ [1cm]
{\textit{\huge{Project type C}}} 
\end{center}~\\ [3 cm]


\begin{center}
\textbf{\huge{Author:}} \\ [1 cm]
\end{center}

\hrule
\begin{itemize}
\item {\Large Gianluca Ruzzi}
\end{itemize}
\hrule

\end{titlepage}








% Define deliverable specific info
\begin{table}[h!]
\begin{tabu} to \textwidth { X[0.3,r,p] X[0.7,l,p] }
\hline
\textbf{Deliverable:} & Project type C\\
\textbf{Title:} & Executing AI algorithm with a coprocessor \\
\textbf{Author:} & Gianluca Ruzzi \\
\textbf{Version:} & 1.0 \\ 
\textbf{Date:} & 10-January-2025 \\
\textbf{Course:} & Performance Evaluation and Application\\
\textbf{Professor:} & Marco Gribaudo \\
\textbf{Copyright:} & Copyright © 2025, Gianluca Ruzzi – All rights reserved \\
\hline
\end{tabu}
\end{table}

\setcounter{page}{2}

\newpage
\addcontentsline{toc}{section}{Table of Contents}
\tableofcontents
\newpage
\addcontentsline{toc}{section}{List of Figures}
\listoffigures
\addcontentsline{toc}{section}{List of Tables}
\listoftables

\clearpage
{\color{Blue}{\section{Introduction}}}
\label{sect:Introduction}
\input{Project document/files/introduction}

\clearpage
{\color{Blue}{\section{Model Architectural Design}}}
\label{sect:Model Architectural Design}
\input{Project document/files/modelarchitecturaldesign}


\clearpage
{\color{Blue}{\section{Solutions}}}
\label{sect:Solutions}
\input{Project document/files/solutions}

\clearpage
{\color{Blue}{\section{Effort Spent}}}
\label{sect:effort}
\input{Project document/files/effort}

\clearpage
\addcontentsline{toc}{section}{References}
\bibliographystyle{plain}
\bibliography{main}
The document is based on the following materials:
\begin{enumerate}
    \item The specification of the RASD and DD assignment of the Software Engineering II
course a.a. 2024/2025
    \item Slides of the course on WeBeep
    \item The RASD document example given during the Software Engineering II
course
\end{enumerate}
The tools used for the realization of this project are:
\begin{enumerate}
    \item \hyperlink{https://staruml.io/}{Star UML}: for the diagrams realization
    \item  \hyperlink{https://moqups.com/it/}{moqups.com}: for the UI mockups
    \item \hyperlink{https://alloytools.org/}{alloy.org}: for the Alloy models
\end{enumerate}


\end{document}